\documentclass[11pt,oneside,final,wide]{mwart} 

\usepackage[utf8]{inputenc} 
\usepackage{polski} 
\usepackage{iwona} 
\usepackage{setspace} 
\usepackage{enumitem} 

\onehalfspacing

\setlength{\parindent}{0pt} 
\setlist{nolistsep}

\setenumerate[2]{label=(\alph*)} 
\setenumerate[3]{label=(\roman*)}

\newcommand{\parno}[1]{\S~#1} 

\newcounter{para}

\newenvironment{para}{
  \refstepcounter{para}
  \par\medskip\pagebreak[3]
	{
  	\centering\textbf{\parno{\arabic{para}}}\par}
  	\nopagebreak\smallskip
  	\ignorespaces
	}
	
	{
  		\par
	}

\newenvironment{alphenumerate}
	{
		\begin{enumerate}[label=(\alph*)]
  	}{
  \end{enumerate}
}

\newcounter{rozdz}
\newlength{\rozdzfil}
\setlength{\rozdzfil}{6cm}

\newcommand{\rozdz}[1]
	{
		\refstepcounter{rozdz}
		\par
  		\vskip 0pt plus \rozdzfil\penalty -200\vskip 0pt plus -\rozdzfil
  		\bigskip
  		{
	  		\centering
			\large\bfseries
			Rozdział~\Roman{rozdz}\par\medskip 
			#1
			\par
		}
		\smallskip
		\nopagebreak
	}

\begin{document}

\begin{center}
	\LARGE
	Regulamin Stowarzyszenia\\
	"Stowarzyszenie przyjaciół Fundacji Hackerspace Kraków"\\
\end{center}

\rozdz{Postanowienia ogólne}

\begin{para}
  \begin{enumerate}
  \item Stowarzyszenie nosi nazwę "Stowarzyszenie przyjaciół Fundacji Hackerspace Kraków" i jest zwane dalej Stowarzyszeniem.
  \end{enumerate}
 \end{para}
\begin{para}
 \begin{enumerate} 
	\item Podstawą działania Stowarzyszenia są przepisy ustawy z dnia 7 kwietnia 1989 r. – Prawo o stowarzyszeniach (tekst jedn. Dz. U. z 2001 r. Nr 79, poz. 855, z późn. zm.) zwanej dalej ustawą, oraz postanowienia niniejszego Regulaminu.
	\item Stowarzyszenie jest stowarzyszeniem zwykłym w rozumieniu ustawy.
 \end{enumerate}
\end{para}

\begin{para}
 Terenem działania Stowarzyszenia jest Rzeczpospolita Polska .
\end{para}

\begin{para}
 Siedzibą Stowarzyszenia jest miasto Kraków.
\end{para}

\rozdz{Cel i środki działania}

\begin{para}
 Celem działania stowarzyszenia jest wspieranie Fundacji Hackerspace Kraków (KRS 0000436707) zwanej dalej Fundacją.
\end{para}

\begin{para}
 Stowarzyszenie realizuje swoje cele poprzez:
 \begin{enumerate}
\item wsparcie finansowe Fundacji,
\item pomoc członków stowarzyszenia podczas warsztatów, prelekcji, konferencji i innych imprez organizowanych przez Fundację,
\item pozyskiwanie sprzętu, narzędzi, materiałów na rzecz Fundacji.
 \end{enumerate}
\end{para}

\rozdz{Władze Stowarzyszenia}

\begin{para}
  Władzami Stowarzyszenia są:
  \begin{enumerate}
  \item Zebranie Członków
  \item Przedstawiciel Stowarzyszenia
  \end{enumerate}
\end{para}

\begin{para}
 \begin{enumerate} 
\item Najwyższą władzą jest Zebranie Członków, które stanowią wszyscy członkowie Stowarzyszenia.
\item Wyróżnia się zebranie sprawozdawczo-wyborcze, na którym dokonuje się wybór nowego Przestawiciela oraz przedstawia się sprawozdanie finansowe i merytoryczne.
 \end{enumerate}
\end{para}

\begin{para}
 \begin{enumerate} 
 \item Zebranie Członków właściwe jest do podejmowania decyzji we wszystkich sprawach Stowarzyszenia, w których regulamin nie określa innych władz.
 \item Do wyłącznych kompetencji Zebrania Członków należą w szczególności:
  \begin{enumerate}
   \item wybór i odwołanie Przedstawiciela stowarzyszenia,
   \item przyjęcie i uchwalanie zmian Regulaminu,
   \item kontrola działań podjętych przez Przedstawiciela,
   \item podjęcie uchwały w sprawie rozwiązania Stowarzyszenia,
   \item ustalanie wysokości składki członkowskiej,
   \item rozpatrywanie skarg członków Stowarzyszenia na działalność Przedstawiciela,
   \item rozpatrywanie odwołań od wykluczenia z Stowarzyszenia.
  \end{enumerate}
 \end{enumerate}
\end{para}

\begin{para}
 \begin{enumerate} 
  \item Uchwały Zebrania Członków Stowarzyszenia o ile regulamin nie stanowi inaczej podejmowane są zwykłą większością głosów, przy obecności co najmniej połowy uprawnionych do głosowania członków.
  \item Uchwała o rozwiązaniu stowarzyszenia, zmiana Regulaminu Stowarzyszenia oraz decyzja o wyborze lub odwołaniu Przestawiciela podejmowana jest bezwzględną większością głosów.
 \end{enumerate}
\end{para}
\begin{para}
  \begin{enumerate}
  \item Każde zebranie wymaga ogłoszenia dwóch terminów w odległości od 30 minut do 2 dni.
  \item W drugim terminie nie jest wymagana obecność połowy uprawnionych do głosowania.
  \item Członkowie uprawnieni do głosowania muszą zostać powiadomieni o obu terminach najpóźniej 7 dni przed pierwszym terminem
  \item Członkowie mogą być powiadomieni za pomocą e-maila, telefonicznie lub listem poleconym.
 \end{enumerate}
\end{para}

\begin{para}
 Zgromadzenie Członków zwołuje Przedstawiciel z własnej inicjatywy lub na wniosek co najmniej 1/5 członków Stowarzyszenia.
\end{para}

\begin{para}
 \begin{enumerate}
  \item Kadencja Przedstawiciela Stowarzyszenia trwa 12 miesięcy.
  \item Przedstawiciel jest zobowiązany do zwołania zebrania sprawozdawczo-wyborczego przed końcem swojej kadencji, na którym zostanie wyłoniony nowy przestawiciel.
 \end{enumerate}
\end{para}

\begin{para}
 \begin{enumerate}
  \item Do kompetencji Przedstawiciela należy:
  \begin{enumerate}
   \item kierowanie bieżącą działalnością Stowarzyszenia,
   \item wykonywanie uchwał Zgromadzenia Członków,
   \item reprezentowanie Stowarzyszenia na zewnątrz i działanie w jego imieniu,
   \item przyjmowanie i wykluczanie ze Stowarzyszenia.
  \end{enumerate}
 \end{enumerate}
\end{para}


\rozdz{Członkostwo}

\begin{para}
 \begin{enumerate}
  \item Członkiem Stowarzyszenia może zostać osoba, który złoży pisemne oświadczenie o przystąpieniu do Stowarzyszenia oraz akceptacji jego Regulaminu.
  \item O przyjęciu/wykluczeniu ze Stowarzyszenia decyduje Przedstawiciel.
  \item Od decyzji o wykluczeniu możliwe jest odwołanie się do Zebrania Członków.
 \end{enumerate}
\end{para}

\begin{para}
 \begin{enumerate}
  \item Do obowiązków członków Stowarzyszenia należy:
  \begin{enumerate}
   \item przestrzeganie Regulaminu Stowarzyszenia,
   \item terminowe uiszczanie składek członkowskich,
   \item realizowanie celów Stowarzyszenia.
  \end{enumerate}
 \end{enumerate}
\end{para}

\rozdz{Środki finansowe Stowarzyszenia}

\begin{para}
 Stowarzyszenie uzyskuje środki na działalność ze składek członkowskich.
\end{para}

\begin{para}
  Środkami finansowymi Stowarzyszenia zarządza Przedstawiciel, zgodnie z obowiązującymi przepisami prawa, niniejszym Regulaminem oraz uchwałami Zgromadzenia Członków.
\end{para}

\rozdz{Postanowienia końcowe}

\begin{para}
 \begin{enumerate}
  \item Stowarzyszenie ulega rozwiązaniu na podstawie uchwały Zgromadzenia Członków o rozwązaniu Stowarzyszania.
  \item Podejmując uchwałę o rozwiązaniu Stowarzyszenia Zgromadzenie Członków określa przeznaczenie pozostałego majątku Stowarzyszenia.
  \item Stowarzyszenie rozwązuje się na skutek likwidacji lub połączenia fundacji z inną fundacją.
 \end{enumerate}
\end{para}

\begin{para}
 \begin{enumerate}
  \item W sprawach nie uregulowanych niniejszym Regulaminem zastosowanie mają przepisy ustawy.
 \end{enumerate}
\end{para}


\end{document}
